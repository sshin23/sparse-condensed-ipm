% \documentclass{IEEEtran4PSCC} % will be used for submission
% \author{
%   \IEEEauthorblockN{Sungho Shin and Mihai Anitescu}
%   \IEEEauthorblockA{
%     Mathematics and Computer Science Division\\
%     Argonne National Laboratory\\
%     Lemont, IL, USA\\
%     sshin@anl.gov, anitescu@mcs.anl.gov}
%   \and
%   \IEEEauthorblockN{François Pacaud}
%   \IEEEauthorblockA{Centre Automatique et Systèmes\\
%     Mines Paris - PSL \\
%     Paris, France\\
%     francois.pacaud@minesparis.psl.eu}
% }
% \psccfooter{%
%         \parbox{\textwidth}{\hrulefill \\ \small{23rd Power Systems Computation Conference} \hfill \begin{minipage}{0.2\textwidth}\centering \vspace*{4pt} \includegraphics[scale=0.06]{PSCC_logo.png}\\\small{PSCC 2024} \end{minipage} \hfill \small{Paris, France --- June 4 -- 7, 2024}}%
% }

\documentclass{IEEEtran} % for ArXiv submission
\author{
  Sungho Shin, François Pacaud, and Mihai Anitescu
  \thanks{
    Sungho Shin and Mihai Anitescu are with Mathematics and Computer Science Division,  Argonne National Laboratory, Lemont, IL, USA. e-mail: \url{sshin@anl.gov}, \url{anitescu@mc.anl.gov}
  }
  \thanks{
    François Pacaud is with Centre Automatique et Systèmes, Mines Paris - PSL, Paris, France. e-mail: \url{francois.pacaud@minesparis.psl.eu}
  }
}



\hyphenation{op-tical net-works semi-conduc-tor}

% Set footer
\makeatletter
\let\old@ps@headings\ps@headings
\let\old@ps@IEEEtitlepagestyle\ps@IEEEtitlepagestyle
\def\psccfooter#1{%
    \def\ps@headings{%
        \old@ps@headings%
        \def\@oddfoot{\strut\hfill#1\hfill\strut}%
        \def\@evenfoot{\strut\hfill#1\hfill\strut}%
    }%
    \def\ps@IEEEtitlepagestyle{%
        \old@ps@IEEEtitlepagestyle%
        \def\@oddfoot{\strut\hfill#1\hfill\strut}%
        \def\@evenfoot{\strut\hfill#1\hfill\strut}%
    }%
    \ps@headings%
}
\makeatother


\usepackage{amsmath,amssymb,amsthm,amsfonts} % assumes amsmath package installed
\usepackage[linktocpage=true,colorlinks=true,linkcolor=blue,citecolor=blue,urlcolor=blue]{hyperref}
% \usepackage[letterpaper,margin=1in]{geometry}

\newtheorem{definition}{Definition}
\newtheorem{assumption}{Assumption}
\newtheorem{theorem}{Theorem}
\newtheorem{conjecture}{Conjecture}
\newtheorem{lemma}{Lemma}
\newtheorem{proposition}{Proposition}
\newtheorem{remark}{Remark}

\newcommand{\bx}{\boldsymbol{x}}
\newcommand{\be}{\boldsymbol{e}}
\newcommand{\blambda}{\boldsymbol{\lambda}}
\newcommand{\bLambda}{\boldsymbol{\Lambda}}
\newcommand{\bu}{\boldsymbol{u}}
\newcommand{\bw}{\boldsymbol{w}}
\newcommand{\by}{\boldsymbol{y}}
\newcommand{\bz}{\boldsymbol{z}}
\newcommand{\bV}{\boldsymbol{V}}
\newcommand{\bX}{\boldsymbol{X}}
\newcommand{\bY}{\boldsymbol{Y}}
\newcommand{\bZ}{\boldsymbol{Z}}
\newcommand{\bv}{\boldsymbol{v}}
\newcommand{\bxi}{\boldsymbol{\xi}}
\newcommand{\bpi}{\boldsymbol{\pi}}
\newcommand{\bphi}{\boldsymbol{\phi}}
\newcommand{\bbeta}{\boldsymbol{\eta}}
\newcommand{\bpsi}{\boldsymbol{\psi}}
\newcommand{\bzeta}{\boldsymbol{\zeta}}
\newcommand{\bmu}{\boldsymbol{\mu}}
\newcommand{\bq}{\boldsymbol{q}}
\newcommand{\bQ}{\boldsymbol{Q}}
\newcommand{\bK}{\boldsymbol{K}}
\newcommand{\bP}{\boldsymbol{P}}
\newcommand{\bS}{\boldsymbol{S}}
\newcommand{\bT}{\boldsymbol{T}}
\newcommand{\bF}{\boldsymbol{F}}
\newcommand{\bG}{\boldsymbol{G}}
\newcommand{\bd}{\boldsymbol{d}}
\newcommand{\bp}{\boldsymbol{p}}
\newcommand{\bff}{\boldsymbol{f}}
\newcommand{\bc}{\boldsymbol{c}}
\newcommand{\bg}{\boldsymbol{g}}
\newcommand{\bh}{\boldsymbol{h}}
\newcommand{\bA}{\boldsymbol{A}}
\newcommand{\bL}{\boldsymbol{L}}
\newcommand{\ba}{\boldsymbol{a}}
\newcommand{\bb}{\boldsymbol{b}}
\newcommand{\bB}{\boldsymbol{B}}
\newcommand{\bC}{\boldsymbol{C}}
\newcommand{\bE}{\boldsymbol{E}}
\newcommand{\bH}{\boldsymbol{H}}
\newcommand{\bR}{\boldsymbol{R}}
\newcommand{\bn}{\boldsymbol{n}}
\newcommand{\bm}{\boldsymbol{m}}
\newcommand{\br}{\boldsymbol{r}}
\newcommand{\bl}{\boldsymbol{l}}
\newcommand{\bI}{\boldsymbol{I}}
\newcommand{\osigma}{\overline{\sigma}}
\newcommand{\usigma}{\underline{\sigma}}
\newcommand{\oosigma}{\overline{\osigma}}
\newcommand{\uusigma}{\underline{\usigma}}
\newcommand{\olambda}{\overline{\lambda}}
\newcommand{\ulambda}{\underline{\lambda}}
\newcommand{\oolambda}{\overline{\olambda}}
\newcommand{\uulambda}{\underline{\ulambda}}
\newcommand{\bzero}{\boldsymbol{0}}
\newcommand{\dist}{\text{\normalfont dist}}
\newcommand{\st}{\mathop{\text{\normalfont s.t.}}}
\newcommand{\diag}{\mathop{\text{\normalfont diag}}}
\newcommand{\amin}{\mathop{\text{\normalfont argmin}}}
\newcommand{\ReH}{\mathop{\text{\normalfont ReH}}}
\newcommand{\bbZ}{\mathbb{Z}}
\newcommand{\cG}{\mathcal{G}}
\newcommand{\cV}{\mathcal{V}}
\newcommand{\cW}{\mathcal{W}}
\newcommand{\cA}{\mathcal{A}}
\newcommand{\cB}{\mathcal{B}}
\newcommand{\cL}{\mathcal{L}}
\newcommand{\cE}{\mathcal{E}}
\newcommand{\cD}{\mathcal{D}}
\newcommand{\cP}{\mathcal{P}}
\newcommand{\cQ}{\mathcal{Q}}
\newcommand{\cK}{\mathcal{K}}
\newcommand{\cM}{\mathcal{M}}
\newcommand{\cN}{\mathcal{N}}
\newcommand{\cI}{\mathcal{I}}
\newcommand{\cJ}{\mathcal{J}}
\newcommand{\cT}{\mathcal{T}}
\newcommand{\interior}{\mathop{\text{\normalfont interior}}}
\newcommand{\relint}{\mathop{\text{\normalfont relint}}}
\newcommand{\vertices}{\mathop{\text{\normalfont vertices}}}
\sloppy
\usepackage{graphicx}

\usepackage{enumitem}
\usepackage{mathtools}
\allowdisplaybreaks
\renewcommand{\theenumi}{(\alph{enumi})}
% \usepackage{parskip}

\usepackage{tikz}
\usepackage{lscape}
\usepackage{multirow}
\usepackage{algorithm}
\usepackage{algorithmic}
\usepackage{cuted}
\setlength\stripsep{3pt plus 1pt minus 1pt}

\usepackage{listings}
\lstset{
basicstyle=\ttfamily\footnotesize, 
columns=fullflexible, % make sure to use fixed-width font, CM typewriter is NOT fixed width
numbers=left, 
numberstyle=\ttfamily,
stepnumber=1,              
numbersep=-2pt, 
numberfirstline=true, 
numberblanklines=true, 
tabsize=4,
lineskip=-1.5pt,
extendedchars=true,
breaklines=true,        
keywordstyle=\color{Blue}\bfseries,
identifierstyle=, % using emph or index keywords
commentstyle=\sffamily\color{OliveGreen},
stringstyle=\color{Maroon},
showstringspaces=false,
showtabs=false,
upquote=false,
texcl=true % interpet comments as LaTeX
}




\lstdefinelanguage{Julia}%
  {morekeywords={abstract,break,case,catch,const,continue,do,else,elseif,%
      end,export,false,for,function,immutable,import,importall,if,in,%
      macro,module,otherwise,quote,return,switch,true,try,type,typealias,%
      using,while},%
   sensitive=true,%
   alsoother={$},%
   morecomment=[l]\#,%
   morecomment=[n]{\#=}{=\#},%
   morestring=[s]{"}{"},%
   morestring=[m]{'}{'},%
}[keywords,comments,strings]%

\lstset{%
  language         = Julia,
  basicstyle       = \ttfamily\small,
  keywordstyle     = \bfseries\color{blue},
  stringstyle      = \color{magenta},
  commentstyle     = \color{ForestGreen},
  showstringspaces = false,
}

\usepackage{cite}